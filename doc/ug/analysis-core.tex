% Copyright (C) 2012,2013,2014,2015,2016 The ESPResSo project
%  
% This file is part of ESPResSo.
%   
% ESPResSo is free software: you can redistribute it and/or modify it
% under the terms of the GNU General Public License as published by the
% Free Software Foundation, either version 3 of the License, or (at your
% option) any later version.
%  
% ESPResSo is distributed in the hope that it will be useful, but
% WITHOUT ANY WARRANTY; without even the implied warranty of
% MERCHANTABILITY or FITNESS FOR A PARTICULAR PURPOSE.  See the GNU
% General Public License for more details.
%  
% You should have received a copy of the GNU General Public License
% along with this program.  If not, see <http://www.gnu.org/licenses/>.
%
\newcommand{\taumax}{\tau_{\mathrm{max}}}
\newcommand{\taumin}{\tau_{\mathrm{min}}}

\chapter{Analysis in the core}
\label{chap:analysis-core}
\index{Analysis in the Core}

Analysis in the core is a new concept introduced in \es since version
3.1.  It was motivated by the fact, that sometimes it is desirable
that the analysis functions do more than just return a value to the
scripting interface.  For some observables it is desirable to be
sampled every few integrations steps. In addition, it should be
possible to pass the observable values to other functions which
compute history-dependent quantities, such as correlation functions.
All this should be done without the need to interrupt the integration
by passing the control to the script level and back, which produces a
significant overhead when performed too often.

Some observables in the core have their corresponding counterparts in
the Tcl observables of the \verb!analyze! command described in
Chapter~\ref{chap:analysis}.  However, only the core-observables can
be used on the fly with the toolbox of the correlator and on the fly
analysis of time series.  Similarly, some special cases of using the
correlator have their redundant counterparts in the analysis in Tcl
(Chapter~\ref{chap:analysis}), but the correlator provides a general
and versatile toolbox which can be used with any implemented
core-observables. The only trick to bridge the gap between Tcl based 
analysis and core analysis is the tclcommand observable that allows
use the return value of arbitrary Tcl functions (also self-written) 
as input for the core analysis. See more below.

\section{Observables}
\index{Observables}

\newescommand{observable}

\subsection{Introduction}
The first step of the core analysis is to tell \es to create an observable.
An observable in the sense of the core analysis can be considered as 
a rule how to compute a certain set of numbers from a given state of the system
or a role how to collect data from other observables. Any observable
is represented as a single array of double values. Any more complex
shape (tensor, complex number, \ldots) must be compatible to this
prerequisite. Every observable however documents the storage order.

Creating an observable means just allocating the corresponding memory, assigning a function 
to compute the observable value and reserving an \var{id} which will be used to refer 
to the observable. 
In addition to the possibility to  print the observable value 
(return the observable value to the script interface), 
the \var{id} of a core-observable can be passed to another analysis function. 
The observable value is computed from the current state of the system
at the moment when it is needed, \ie when requested explicitly by the 
user calling the \verb!observable print! function or when requested
automatically by some other analysis function. Updating is an orthogonal 
concept: Observables that collect data over time (e.g. the average observable)
need to be updated regularly, even though their current value is not
of interest. 

Not all observables are implemented in parallel. When performing a parallel
computation, too frequent updates to observables which are not implemented
in parallel may produce a significant slowdown.

\subsection{Creating an observable}
To create a new observable, use
\begin{essyntax}
  observable new \var{name} \opt{\var{parameters+}}
\end{essyntax}
  
Upon this call, \es allocates the necessary amount of memory and returns 
an integer \var{id} which will be used later to refer to the observable. 
The parameter \var{name} and further arguments have to correspond to one of the
observables described below.

\subsubsection{Available observables}
Currently the following observables are implemented.
Particle specifications (see section~\ref{sec:PartSpec} below)
define a group of particles, from which the observable should be calculated. 
They are generic to all observables and are described after the list of observables.

\todo{Missing descriptions of parameters of several observables}
Here are the observables, that only depend on the current state of the simulation
system:
  \begin{itemize}
    \item \lit{particle_positions} \var{particle\_specifications}\\
          Positions of the particles, in the format 
          $x_1,\ y_1,\ z_1,\ x_2,\ y_2,\ z_2,\ \dots\ x_n,\ y_n,\ z_n$. 
          The particles are ordered ascending according to their ids.
    \item \lit{particle_velocities} \var{particle\_specifications}\\
          Velocities of the particles, in the format\\ 
          $v^x_1,\ v^y_1,\ v^z_1,\ v^x_2,\ v^y_2,\ v^z_2,\ 
          \dots\ v^x_n,\ v^y_n,\ v^z_n$. 
          The particles are ordered ascending according to their ids.
    \item \lit{particle_body_velocities} \var{particle\_specifications}\\
          Velocities of the particles in the body frame, in the format\\ 
          $v^x_1,\ v^y_1,\ v^z_1,\ v^x_2,\ v^y_2,\ v^z_2,\ 
          \dots\ v^x_n,\ v^y_n,\ v^z_n$. 
          The particles are ordered ascending according to their ids. This
          command only produces a meaningful result when \texttt{ROTATIONS}
          is compiled in.
    \item \lit{particle_forces} \var{particle\_specifications}\\
          Forces on the particles, in the format\\ 
          $f^x_1,\ f^y_1,\ f^z_1,\ f^x_2,\ f^y_2,\ f^z_2,\ 
          \dots\ f^x_n,\ f^y_n,\ f^z_n$. 
          The particles are ordered ascending according to their ids.
    \item \lit{particle_angular_momentum} \var{particle\_specifications}\\
          Angular momenta (omega) of the particles, in the format\\ 
          $\omega^x_1,\ \omega^y_1,\ \omega^z_1,\ \omega^x_2,\ \omega^y_2,\ \omega^z_2,\ 
          \dots\ \omega^x_n,\ \omega^y_n,\ \omega^z_n$. 
          The particles are ordered ascending according to their ids and the 
          angular velocity/momentum is specified in the laboratory frame.
    \item \lit{particle_body_angular_momentum} \var{particle\_specifications}\\
          Angular momenta (omega) of the particles, in the format\\ 
          $\omega^x_1,\ \omega^y_1,\ \omega^z_1,\ \omega^x_2,\ \omega^y_2,\ \omega^z_2,\ 
          \dots\ \omega^x_n,\ \omega^y_n,\ \omega^z_n$. 
          The particles are ordered ascending according to their ids and the 
          angular velocity/momentum is specified in the body (co-rotating) frame.
    \item \lit{com_position} \var{particle\_specifications} \opt{blocked \var{size}}\\
          Position of the centre of mass. 
          If \lit{blocked \var{size}} is specified,
          the particles are subdivided into blocks of size \var{size} and the centre
          of mass position is calculated for each block separately.
    \item \lit{com_velocity} \var{particle\_specifications} \opt{blocked \var{size}}\\
          Velocity of the centre of mass. 
          If \lit{blocked \var{size}} is specified,
          the particles are subdivided into blocks of size \var{size} and the centre
          of mass velocity is calculated for each block separately.
    \item \lit{com_force} \var{particle\_specifications} \opt{blocked \var{size}}\\
          Total force on the specified particles. 
          If \lit{blocked \var{size}} is specified,
          the particles are subdivided into blocks of size \var{size} and the total
          force is calculated for each block separately.
    \item \lit{stress_tensor} \\
          The stress tensor. It only works with all particles. 
          It is returned as a 9-dimensional array:\\
          $ \{\ \sigma_{xx},\ \sigma_{xy},\ \sigma_{xz},\ \sigma_{yx},\ \sigma_{yy},\
          \sigma_{yz},\ \sigma_{zx},\ \sigma_{zy},\ \sigma_{zz}\ \} $
    \item \lit{stress_tensor_acf_obs} \\
          The observable for computation of the Stress tensor autocorrelation function. 
          Similarly to the stress tensor, it only works with all particles.
          \todo{any suggestion for a more suitable name?}
          It is returned as a 6-dimensional array:\\
          $ \{\ \sigma_{xy},\ \sigma_{yz},\ \sigma_{zx},\ 
          ( \sigma_{xx} - \sigma_{yy}),\ 
          ( \sigma_{xx} - \sigma_{zz}),\ 
          ( \sigma_{yy} - \sigma_{zz})\  
          \} $ \\
          where $\sigma_{ij}$ are the components of the stress tensor.
    \item \lit{particle_currents} \var{particle\_specifications}\\
          Electric currents due to individual particles. For a particle $i$:
          $j_i^x = q_i v_i^x / \Delta t$
          where $\Delta t$ is the simulation time step.
          Required feature: \required{ELECTROSTATICS}
    \item \lit{currents} \var{particle\_specifications}\\
          Electric currents averaged over all particles: 
          $j^x = \sum_i q_i v_i^x / \Delta t$
          where $\Delta t$ is the simulation time step.
          Required feature: \required{ELECTROSTATICS}
    \item \lit{dipole_moment} \var{particle\_specifications}\\
          The dipole moment of the specified group of particles:
          $\mu^x = \sum_i q_i r_i^x$
          Required feature: \required{ELECTROSTATICS}
    \item \lit{com_dipole_moment} \var{particle\_specifications}\\
          Total dipole moment of the specified group of particles:
          $\mu^x = \sum_i d_i^x$
          where $d_i^x$ is an intrinsic dipole moment of the
          individual $i$-th particle.
          Required feature: \required{DIPOLES}
%    \item \lit{structure_factor} \var{order} \var{particle\_specifications}\\
%          Structure factor of the specified particles. Meaning of the parameter
%          \var{order} and the output formate are the same as 
%          \lit{analyze structurefactor} described in section~\ref{sed:analysis}.
%          \todo{implementation not finished}
% FIXME When changing the observable concept, the structure factor was not fully ported. Looks
% like nobody complained until now.
    \item \lit{interacts_with} \var{particle\_specifications1} \var{particle\_specifications2} \var{cutoff} \\
          For each particle belonging to \var{particle\_specifications1} 
          the observable is unity if a neighbour of a type from 
          \var{particle\_specifications2} is found within the distance 
          defined by the \var{cutoff}. If no such neighbour is found, the 
          observable is zero. The observable has one dimension per each 
          particle of \var{particle\_specifications_1}
    \item \lit{density_profile}  \var{particle\_specifications} \var{profile\_specifications} \\
          Compute the density profile within the specified cube.
          For profile specifications, see section~\ref{sec:DensProfSpec}.
\item \lit{force_density_profile}  \var{particle\_specifications} \var{profile\_specifications} \\
          Compute the force density profile within the specified cube.
          For profile specifications, see section~\ref{sec:DensProfSpec}.
    \item \lit{lb_velocity_profile}  \var{particle\_specifications} \var{profile\_specifications} \\
          Compute the Lattice-Boltzmann velocity profile within the specified cube.
          For profile specifications, see section~\ref{sec:DensProfSpec}.

    \item \lit{flux_density_profile}  \var{particle\_specifications} \var{profile\_specifications} \\
          Compute the flux density within the specified cube.
          For profile specifications, see section~\ref{sec:DensProfSpec}.
    \item \lit{radial_density_profile}
        Compute the density profile in cylindrical coordinates.
          For profile specifications, see section~\ref{sec:DensProfSpec}.
    \item \lit{radial_flux_density_profile} \\
        Compute the flux density profile in cylindrical coordinates.
          For profile specifications, see section~\ref{sec:DensProfSpec}.
    \item \lit{lb_radial_velocity_profile} \\
        Compute the Lattice-Boltzmann velocity profile in cylindrical coordinates.
          For profile specifications, see section~\ref{sec:DensProfSpec}.
    \item \lit{rdf} \var{type\_list type\_list [r\_min [r\_max [n\_bins]]]}\\
        Compute the radial distribution function, see \ref{analyze:rdf}.
    \item \lit{structure_factor} \var{order}\\
        Compute the structure factor. Remember it scales as order$^3$, see \ref{analyze:structurefactor}.
	\item \lit{radial_density_distribution} \var{type <type> minr <minr>}\verb!\! \\
		\var{maxr <maxr> rbins <rbins> ( start\_point <X> <Y> <Z>}\verb!\!\\
		\var{end\_point <X> <Y> <Z> | id\_start\_point <id> id\_end\_point <id> )} \\
		Computes the radial density distribution for particles of the given type around the axis
		given either as fixed positions or as particle ids. The binning is done
		between \verb!minr! and \verb!maxr! with \verb!rbins!. 
	\item \lit{spatial_polymer_property} \var{(ids <id\_list> | type <type>) N
		<Npoly>} \\
		Calculates the mean charge along \verb!Npoly! weak polyelectrolytes of the \emph{same}
		length. The particles can be specified as a list of particle ids or through
		the particle type. The result will be the average charge during the
		simulation in dependence of the monomer ranking number, \ie{} its position on
		the chain. 
	\item \lit{persistence_length} \var{ids <id\_list> max\_d <max\_d> cut\_off
		<cut\_off>} \\
		Calculates the persistence length based on the bond vector angle correlation of the given polymer specified through the
		\verb!id_list!. \verb!max_d! is the maximum distance (in terms of particles) for
		which the correlation is computed. With \verb!cut_off! the number of
		particles at the polymer ends that are ignored for the calculation, can be
		specified. The observable will not calculate the actual persistence length
		but the correlation function of the bond angles. 
		\begin{equation}
			\mathrm{Obs}(i) =\mathrm{Norm\_const} \langle \sum_j \frac{(\vec{r}_j - \vec{r}_{j+1})
			(\vec{r}_{j+i}-\vec{r}_{j+i+1})}{|\vec{r}_j-\vec{r}_{j+1}||\vec{r}_{i+j}-\vec{r}_{i+j+1}|}\rangle 
		\end{equation}
	\item \lit{polymer_pair_correlation} \var{ids <id\_list> maxr <maxr> minr <minr>
		k <k>}\verb!\!\\
		\var{rbins <rbins> N <Npoly> poly\_len <poly\_len>} \\
		Computes the pair correlation of particles on a polymer chain given through
		the \verb!id_list!, here \verb!k! is the distance (in terms of particles) between
		the mononers on the chain for which the pair correlation is computed. This
		is averaged over the whole chain and all the given chains. The number of
		polymers for which this distribution is computed has to be given as
		\verb!Npoly! and their length through \verb!poly_len!. The distribution is
		calculated for distances between \verb!minr! and \verb!maxr! with
		\verb!rbins!. 
		\begin{equation}
			\mathrm{Obs}(r) = \mathrm{Norm\_const} \langle \sum_{j=0,
			j+=k}^{j<\mathrm{poly\_len} - k} \delta(r - |\vec{r}_j -
			\vec{r}_{j+k}|)\rangle
		\end{equation}
      \end{itemize}
The tclcommand observable is a helpful tool, that allows to make the 
analysis framework much more versatile, by allowing
the evaluation of arbitrary tcl commands.
  \begin{itemize}
    \item \lit{tclcommand \var{dimQ} \var{command}} \\
        An arbitrary Tcl function that returns a list of floating point numbers of fixed size
        \var{dimQ} can be specified. Although its execution might be slow, it allows 
        to prototype new observables without a lot of trouble.
        Many existing analysis commands can be made to cooperate with the core analysis that way.
  \end{itemize}
The following commands allow to collect data automatically over time
once their autoupdate feature is enabled.

  \begin{itemize}
    \item \lit{average \var{ref}} \\
      The running average of the reference observable with id \var{ref}. 
      It can be resetted by \lit{observable \var{no} reset}
  \end{itemize}
\subsection{Printing an observable}
\begin{essyntax}
observable \var{id} print \opt{no_calculation} \opt{formatted}
\end{essyntax}
Prints the value of the observable with a given $id$. If the observable
refers to the current state of the system, its value is updated before printing, except
if \lit{no_calculation} is given.
\todo{Formatted printing is not fully supported yet.}

\subsection{Checkpointing observables}
\begin{essyntax}
observable \var{id} write_checkpoint \var{filename} \opt{binary}
\end{essyntax}
Writes the current state of the observable to \lit{filename}.
If \lit{binary} is given then it does so in binary form. This is useful
to save the state of statful observables as the average.
This checkpoing mechanism is not portable, rereading is only guaranteed to work
on the same system with the same Espresso binary.

\begin{essyntax}
observable \var{id} read_checkpoint \var{filename} \opt{binary}
\end{essyntax}
Reads the state of the observable from a checkpoint file. The observable
has to be configured beforehand in the script in exactly the same way
as it was when the checkpoint was written. That means that the configuration
of the observable -- including possible particle and profile specifications -- is not saved in the file
but has to be provided by the user. Please be aware that the value of observables that refer to the
current state of the system are overwritten by the \lit{print} command except if the option \lit{no_calculation} is given.

\subsection{Passing an observable to an analysis function}
Currently the only analysis function which uses the core observables
is the correlator (section~\ref{sec:Correlations}).

\subsection{Deleting an observable to an analysis function}
\begin{essyntax}
observable \var{id} delete
\end{essyntax}
Deletes the observable, \ie frees the allocated memory
and makes the $id$ free for a new observable.
\todo{Does not work yet}


\subsection{Particle specifications}
\label{sec:PartSpec}
You can specify from which particles the observable should be computed in one of 
the following ways. In all cases, particle specifications refer to the current
state of espresso. Any later changes to particles (additions, deletions, changes
of types) will not be automatically reflected in the observable.
  \begin{itemize}
    \item \lit{all} \\
          Requests observable calculation based on all particles in the system.
    \item \lit{types} \var{ type\_list } \\
          Restricts observable calculation to a given particle type(s). The type
	  list is a tcl list of existing particle types.
    \item \lit{id} \var{ id\_list } \\
          Restricts observable calculation to a given list of particle id(s). The id 
	  list is a tcl list of existing particle ids.
% The following two particle specifications have not been implemented yet
%    \item \lit{blocks} \var{ m } \\
%          From an $n$-dimensional observable craeates an $n/m$-dimensional one by 
%	  averaging over $m$ neighbouring entries in the data array.
%      \todo{Not fully supported yet!}
%    \item \lit{strides} \var{ m } \\
%          From an $n$-dimensional observable craeates an $n/m$-dimensional one by 
%	  averaging over entries which are separated by $m$ in the data array.
%      \todo{Not fully supported yet!}
  \end{itemize}

\subsection{Profile specifications}
\label{sec:DensProfSpec}
Profiles are specified by giving the spacial area that is to be profiled and
the number of bins in each spacial direction. The area to be analyzed is characterized
by \var{minx}/\var{maxx} \var{miny}/\var{maxy} and \var{minz}/\var{maxz}. The defaults
correspond to the box size when the observable is created.
The bin size in each direction defaults to 1, and can be change with the parameter 
\var{xbins}/\var{ybins}/\var{zbins}. Changing one, two or three of them to 
a value $>1$ with thus create a one-, two- or three-dimensional map of the desired quantity.
The full syntax thus reads as:
\begin{essyntax}
    observable new needs\_profile\_specs \opt{\var{other\_parameters}} 
    \opt{ minx \var{minx}  }
    \opt{ maxx \var{maxx}  }
    \opt{ miny \var{miny}  }
    \opt{ maxy \var{maxy}  }
    \opt{ minz \var{minz}  }
    \opt{ maxz \var{maxz}  }
    \opt{ xbins \var{xbins}  }
    \opt{ ybins \var{ybins}  }
    \opt{ zbins \var{zbins}  }
\end{essyntax}

Radial profiles allow to do the same as usual profiles, except the coordinate system is a cylindrical
one and the binning is done in the cylindrical coordinates (defined with the axis in z-direction). 
This is very helpful if the symmetry of the
system is cylindrical. The spacial are is characterized by a center (default to the center of the box)
a maximum radial position \var{maxr} (defaults to the smaller value of the box lengths in x and y directions)
and a minimum and maximum value of $z$. It is possible to also resolve different polar angles, thus using it
as a full 3D mapping tool, but this will only rarely be used. The full syntax is:
\begin{essyntax}
    observable new needs\_radial\_profile\_specs \opt{\var{other\_parameters}} 
    \opt{ \var{center} <cx> <cy> <cx> }
    \opt{ maxr \var{maxr}  }
    \opt{ minz \var{minz}  }
    \opt{ maxz \var{maxz}  }
    \opt{ rbins \var{rbins}  }
    \opt{ phibins \var{phibins}  }
    \opt{ zbins \var{zbins}  }
\end{essyntax}


\section{Correlations}
\index{Correlations}
\label{sec:Correlations}

\subsection{Introduction}
Time correlation functions are ubiquitous in statistical mechanics and molecular
simulations when dynamical properties of many-body systems are concerned.
A prominent example is the velocity autocorrelation function,  
$ \left< \mathbf{v}(t) \cdot \mathbf{v}(t+\tau) \right> $ 
which is used in the Green-Kubo relations.
In general, time correlation functions are of the form
\begin{equation}
C(\tau) = \left<A\left(t\right) \otimes B\left(t+\tau\right)\right>\,,
\label{eq:corr.def}
\end{equation}
where $t$ is time, $\tau$ is the lag time (time difference) between 
the measurements of (vector) observables $A$ and $B$, and $\otimes$ is an
operator which produces the vector quantity $C$ from $A$ and $B$. 
The ensemble average $\left< \cdot \right>$ is taken over all time origins~$t$. 
Correlation functions describing dynamics of large and complex molecules 
such as polymers span many orders of magnitude, ranging from MD time step
up to the total simulation time. 

\es uses a fast correlation algorithm (see section~\ref{sec:multipleTau})
which enables efficient computation of correlation functions spanning
many orders of magnitude in the lag time. 

The generic correlation interface of \es may process either observables
defined in the kernel, or data which it reads from an external file
or values entered through the scripting interface. 
\todo{Processing data from Tcl input or from input files is not
  fully supported yet.}
Thus, apart from
data processing on the fly, it can also be used as an efficient correlator
for stored data. In all cases it produces a matrix of 
$n+2$ columns. The first two columns are the values of lag times $\tau$ and 
the number of samples taken for a particular value of $\tau$. The
remaining ones are the elements of the $n$-dimensional vector $C(\tau)$.

The \lit{uwerr} command for computing averages and error estimates 
of a time series of observables relies on estimates of autocorrelation
functions and the respective autocorrelation times.
The correlator provides the same functionality as a by-product of computing
the correlation function (see section~\ref{ssec:CorrError}.

An example of the usage of observables and correlations is provided 
in the script \lit{correlation.tcl} in the samples directory.


\subsection{Creating a correlation}
\newescommand{correlation}

Correlation first has to be defined by saying which observables 
are to be correlated, what should be the correlation operation, sampling
frequency, etc. When a correlation is defined, its id is returned which is
used further to do other operations with the correlation.
The correlation can be either updated automatically on the fly without
direct user intervention, or by an explicit user call for an update.

\begin{essyntax}
correlation new obs1 \var{id1} \opt{obs2 \var{id2}} corr_operation \var{operation} dt \var{dt} tau_max \var{tau\_max} \opt{tau_lin \var{tau\_lin}} \opt{compress1 \var{name} \opt{compress2 \var{name}} }
\end{essyntax}

Defines a new correlation and returns an integer
\var{id} which has been assigned to it. 
Its further arguments are described below.

\begin{arguments}
\item \lit{obs1} and \lit{obs2} \\ 
  are ids of the observables A and B that are to correlated. The ids have to refer to existing 
  observables which have been previously defined by the \lit{observable} command.
  Some observables are already implemented, and others can be easily added. This can be done
  with very limited \es{} knowledge just by following the implementations that are already
  in. If \lit{obs2} is omitted, autocorrelation of \lit{obs1} is calculated by default.
\item \lit{corr_operation} \\
  The operation that is performed on $A(t)$ and $B(t+\tau)$ to obtain $C(\tau)$. 
  The following operations are currently is available:
  \begin{itemize}
    \item \lit{scalar_product} \\
    Scalar product of $A$ and $B$, \ie $C=\sum\limits_{i} A_i B_i$
    \item \lit{componentwise_product} \\
    Comnponentwise product of $A$ and $B$, \ie $C_i = A_i B_i$
    \item \lit{square_distance_componentwise} \\
    Each component of the correlation vector is the square of the difference between the 
    corresponding components of the observables, \ie $C_i = (A_i-B_i)^2$. 
    Example: when $A$ is \lit{particle_positions}, it produces the mean square displacement
    (for each component separately).
    \item \lit{tensor_product} \\
    Tensor product of $A$ and $B$, \ie $C_{i \cdot l_B + j} = A_i B_j$, with $l_B$ the length of $B$.
    \item \lit{complex_conjugate_product}
    \todo{Complex conjugate product must be defined.}
    \item \lit{fcs_acf} \var{w_x} \var{w_y} \var{w_z} \\
    Fluorescence Correlation Spectroscopy (FCS) autocorrelation function, \ie
    \begin{equation}
    G_i(\tau) = \frac{1}{N} \Bigl< \exp \Bigl( - \frac{\Delta x_i^2(\tau) }{w_x^2} - \frac{\Delta y_i^2(\tau)}{w_y^2} - \frac{\Delta z_i^2(\tau)}{w_z^2} \Bigr) \Bigr>\,,
    \label{eq:Gtau}
    \end{equation}
    where $\Delta x_i^2(\tau) = \bigl(x_i(0) - x_i(\tau) \bigr)^2$ is the square discplacement 
    of particle $i$ in the $x$ direction, and $w_x$ is the beam waist of the intensity profile 
    of the exciting laser beam,
    \begin{equation}
    W(x,y,z) = I_0 \exp \Bigl( - \frac{2x^2}{w_x^2} - \frac{2y^2}{w_y^2} - \frac{2z^2}{w_z^2} \Bigr)\,.
    \end{equation}
    Equation~\eqref{eq:Gtau} is a generalization of the formula presented by 
    H\"ofling \etal~\cite{hofling11a}. For more information, see references therein.
    Per each 3 dimensions of the observable, one dimension of the correlation output
    is produced. If \lit{fcs_acf} is used with other observables than 
    \lit{particle_positions}, the physical meaning of the result is unclear.
    %\item List them here! \todo{write the list}
  \end{itemize}
\item \lit{dt} \\
  The time interval of sampling data points. When autoupdate is used, \var{dt} has
  to be a multiple of timestep. It is also used to produce time axis in real units.
  \textit{Warning: if \var{dt} is close to the timestep, autoupdate is strongly recommended.
  Otherwise cpu time is wasted on passing the control between the script and kernel.}
\item \lit{tau_max} \\
  This is the maximum value of $\tau$ for which the correlation should be computed.
  \textit{Warning: Unless you are using the multiple tau correlator, choosing \var{tau\_max}
  of more than 100\var{dt} will result in a huge computational overhead.
  In a multiple tau correlator with reasonable parameters, 
  \var{tau\_max} can span the entire simulation without
  too much additional cpu time.}
\item \lit{tau_lin} \\
  The number of data-points for which the results are linearly spaced
  in tau.  This is a parameter of the multiple tau correlator. If you
  want to use it, make sure that you know how it works. By default, it
  is set equal to \var{tau\_max} which results in the trivial linear
  correlator. By setting \var{tau\_lin} $<$ \var{tau\_max} the multiple
  tau correlator is switched on. In many cases, \var{tau\_lin}=16 is a
  good choice but this may strongly depend on the observables you are
  correlating.  For more information, we recommend to read
  Ref.~\cite{ramirez10a} or to perform your own tests.
\item \lit{compress1} and \lit{compress2} \\
  Are functions used to compress the data when going to the next level
  of the multiple tau correlator. Different compression functions for
  different observables can be specified if desired, otherwise the
  same function is used for both.  Default is \lit{discard} which
  takes one of the observable values and discards the other one. This
  is safe for all observables but produces poor statistics in the
  tail. For some observables, \lit{linear} compression can be used
  which makes an average of two neighbouring values but produces
  systematic errors.  Depending on the observable, the systematic
  error can be anything between harmless and disastrous. For more
  information, we recommend to read Ref.~\cite{ramirez10a} or to
  perform your own tests.
\end{arguments}

\subsection{Inquiring about already existing correlations}
\begin{essyntax}
\variant{1} correlation 
\variant{2} correlation n_corr
\end{essyntax}

Variant \variant{1} returns a tcl list of the defined correlations
including their parameters.  \todo{Maybe not all parameters are
  printed.}  

Variant \variant{2} returns the number of currently
defined correlations.  
  
\subsection{Collecting time series data for the correlation}

\begin{essyntax}
\variant{1} correlation \var{id} autoupdate \{ start | stop\} 
\variant{2} correlation \var{id} update 
\variant{3} correlation \var{id} finalize
\end{essyntax}

Variant \variant{1} is the recommended way of updating the correlations.
By specifying \lit{start} or \lit{stop} it starts or stops automatically
updating the correlation estimates. The automatic updates are done
within the integration loop without further user intervention.
The update frequency is adjusted based on the value of \var{dt} 
provided when defining the correlation. Note that autoupdate has to be 
started \underline{after} setting the sim-time (e.g. after \var{setmd time 0}).

Variant \variant{2} is an explicit call for an instantaneous 
update of the correlation estimates, using the current system
state. It is only possible to use \variant{2} if the correlation
is not being autoupdated. However, it is possible to use it
after autoupdate has been stopped. When updating by an explicit
call, \es does not check if the lag time between two updates
corresponds the value of \var{dt} specified when creating the 
correlation.

Variant \variant{3} correlates all data from history which are left in
the buffers. Once this has been done, the history is lost and no
further updates are possible. 
When a new observable value is passed to a correlation,
level 0 of the compression buffers of the multiple tau
correlator (see section~\ref{sec:multipleTau} for details) 
is updated immediately. Higher levels are
updated only when the lower level buffers are filled and
there is a need to push some values one level up. When the
updating is stopped, a number of observable values have not
reached the higher level, especially when \var{tau\_max} 
is comparable to the total simulation time
and if there are many compression levels. In such case,
variant \variant{3} is very useful. If \var{tau\_max} 
is much shorter, it does not have a big effect.


\subsection{Printing out the correlation and related quantities}
\label{ssec:CorrError}
\begin{essyntax}
\variant{1} correlation \var{id} write_to_file \var{filename}
\variant{2} correlation \var{id} print 
\variant{3a} correlation \var{id} print \opt{ average1 | variance1 | correlation_time } 
\variant{3b} correlation \var{id} print \opt{ average_errorbars }
\end{essyntax}

Variant \variant{1} writes the current status of the correlation
estimate to the specified filename. If the file exists, its contents will
be overwritten.

\minisec{Output format} 
The output looks as follows:
\begin{code}
tau1 n_samples C1 C2 ... Cn
tau2 n_samples C1 C2 ... Cn
\end{code}
Where each line corresponds to a given value of \lit{tau}, \lit{n_samples} is the number
of samples which contributed to the correlation at this level and $C_i$ are the individual
components of the correlation.

Variant \variant{2} returns the current status of the correlation
estimate as a Tcl variable. 
\minisec{Output format} 
The output looks as follows:
\begin{code}
{ tau1 n_samples C1 C2 ... Cn }
{ tau2 n_samples C1 C2 ... Cn }
\end{code}

Variants \variant{3a} and \variant{3b} return the corresponding
estimate of the statistical property as a Tcl variable.             \\
\lit{average1} prints the average of observable1.                   \\
\lit{variance1} prints the variance of observable1.                 \\
\lit{correlation_time} prints the estimate of the correlation time. \\
\lit{average_errorbars} prints the estimate of the error of the average based
on the method according to~\cite{wolff04a} (same as 
used by the \lit{uwerr} command). 


\subsection{The correlation algorithm: multiple tau correlator}
\label{sec:multipleTau}
\index{Multiple tau correlator}

Here we briefly describe the multiple tau correlator which is implemented in \es.
For a more detailed description and discussion of its behaviour with respect to
statistical and systematic errors, please read the cited literature.
This type of correlator has been in use for years in the analysis of
dynamic light scattering~\cite{schatzel88a}. About a decade later it found its way
to the Fluorescence Correlation Spectroscopy (FCS)~\cite{magatti01a}.
The book of Frenkel and Smit~\cite{frenkel02b} describes its application
for the special case of the velocity autocorrelation function.

\begin{figure}[ht]
\begin{center} 
\includegraphics[width=0.9\textwidth]{figures/correlator_scheme}
\end{center} 
\caption{Schematic representation of buffers in the correlator.}
\label{fig:dataSet}
\end{figure}

Let us consider a set of $N$ observable values as schematically shown
in figure~\ref{fig:dataSet}, where a value of index $i$ was measured
in time $i\delta t$. We are interested in computing the correlation 
function according to equation~\eqref{eq:corr.def} for a range lag times
$\tau = (i-j)\delta t$ between the measurements $i$ and $j$.
To simplify the notation, we further drop $\delta t$
when referring to observables and lag times. 

The trivial implementation takes all possible pairs of values
corresponding to lag times $\tau \in [\taumin:\taumax]$. 
Without loss of generality, let us further consider $\taumin=0$.
The computational effort for such an algorithm scales
as ${\cal O} \bigl(\taumax^2\bigr)$.
As a rule of thumb, this is feasible if $\taumax < 10^3$.
The multiple tau correlator provides a solution to compute the
correlation functions for arbitrary range of the lag times by
coarse-graining the high $\tau$ values. It applies the naive algorithm
to a relatively small range of lag times $\tau \in [0:p-1]$. This we refer
to as compression level 0. To compute the correlations for lag times
$\tau \in [p:2(p-1)]$, the original data are first coarse-grained, so
that $m$ values of the original data are compressed to produce a single
data point in the higher compression level. Thus the lag time between
the neighbouring values in the higher compression level increases
by a factor of $m$, while the number of stored values decreases by
the same factor and the number of correlation operations at this level
reduces by a factor of $m^2$. Correlations for lag times 
$\tau \in [2p:4(p-1)]$ are computed at compression level 2, which is created
in an analogous manner from level 1. This can continue hierarchically
up to an arbitrary level for which enough data is available. Due to the
hierarchical reduction of the data, the algorithm scales as 
${\cal O} \bigl( p^2 \log(\taumax) \bigr)$. Thus an additional order
of magnitude in $\taumax$ costs just a constant extra effort.

The speedup is gained at the expense of statistical accuracy.
The loss of accuracy occurs at the compression step.
In principle one can use any value of $m$ and $p$ to tune the algorithm
performance. However, it turns out that using a high $m$ dilutes the
data at high $\tau$. Therefore $m=2$ is hard-coded in the \es correlator
and cannot be modified by user. The value of $p$ remains an adjustable
parameter which can be modified by user by setting \lit{tau_lin}
when defining a correlation. In general, one should choose $p \gg m$
to avoid loss of statistical accuracy. Choosing $p=16$ seems to be
safe but it may depend on the properties of the analyzed
correlation functions. A detailed analysis has been performed
in Ref.~\cite{ramirez10a}.

The choice of the compression function also influences the statistical
accuracy and can even lead to systematic errors. The default compression 
function is \lit{discard2} which discards the second for the compressed 
values and pushes the first one to the higher level. This is robust and 
can be applied universally to any combination of observables and
correlation operation. On the other hand, it reduces the
statistical accuracy as the compression level increases.
In many cases, the \lit{average} compression operation
can be applied, which averages the two neighbouring values
and the average then enters the higher level, preserving
almost the full statistical accuracy of the original data. 
In general, if averaging can be safely used or not, depends on the 
properties of the difference
\begin{equation} 
\frac{1}{2} (A_i \otimes B_{i+p} + A_{i+1} \otimes B_{i+p+1} ) - 
\frac{1}{2} (A_i + A_{i+1} ) \otimes \frac{1}{2} (B_{i+p} +  B_{i+p+1})
\label{eq:difference}
\end{equation} 
For example in the case of velocity autocorrelation function, the
above-mentioned difference has a small value and a random sign, \ie\ 
different contributions cancel each other. On the other hand, in the
of the case of mean square displacement the difference is always positive,
resulting in a non-negligible systematic error. A more general
discussion is presented in Ref.~\cite{ramirez10a}.


\subsection{Checkpointing the correlator}
It is possible to checkpoint the correlator. Thereby the data is written
directly to a file. It may be usefull to write to a binary file, as this
preserves the full bit-value of the variables, whereas the text file 
representation has a lower accuracy.

\begin{essyntax}
\variant{1} correlation \var{id} write_checkpoint_binary \var{filename}
\variant{2} correlation \var{id} write_checkpoint_ascii \var{filename}
\end{essyntax}

In order to load a checkpoint, the correlator has to be initialized. Therefore
the observable(s) have to be created. Make sure that the correlator is exactly
initilized as it was when the checkpoint was created. If this is not
fullfilled, and e.g. the size of an observable has changed, loading the
checkpoint failes.
\begin{essyntax}
\variant{1} correlation \var{id} read_checkpoint_binary \var{filename}
\variant{2} correlation \var{id} read_checkpoint_ascii \var{filename}
\end{essyntax}
Depending on whether the checkpoint was written as binary or as text, the
corresponding variant for reading the checkpoint has to be used.

An simple example for checkpointing:
\begin{tclcode}
set pp  [observable new particle_positions all]
set cor1 [correlation new obs1 $pp corr_operation square_distance_componentwise \
dt 0.01 tau_max 1000 tau_lin 16]
integrate 1000
correlation $cor1 write_checkpoint_binary "cor1.bin"
\end{tclcode}
And then to continue the simulation:
\begin{tclcode}
set pp  [observable new particle_positions all]
set cor1 [correlation new obs1 $pp corr_operation square_distance_componentwise \
dt 0.01 tau_max 1000 tau_lin 16]
correlation $cor1 read_checkpoint_binary "cor1.bin"
\end{tclcode}
